\documentclass[12pt]{article}
\usepackage[utf8x]{inputenc}
\usepackage{authblk}
\usepackage{color,soul}
\usepackage[square,sort,comma,numbers]{natbib}
\usepackage{graphicx}

\title{DNA Sequence Classification for detection of plasmid fragments}
\author{Sharanya Kamath}
\author{Mehnaz Yunus}
\author{Shashidhar Koolagudi}
\affil{National Institute of Technology, Karnataka}

\renewcommand\Authands{ and }
\begin{document}
\maketitle

\begin{abstract}
DNA classification is the problem of identifying the functionality of genes using only the sequence information (ATGTGT...) automatically. Plasmids are circular or linear double-stranded DNA molecules which are capable of autonomous replication and are transferable between different bacteria. Computational methods for prediction of genomic elements such as genes are significantly different for chromosomes and plasmids, hence raising the need for separation of chromosomal from plasmid sequences in a metagenome.

We developed a machine learning model for discrimination between plasmid-derived and chromosome-derived sequences.

\end{abstract}

\section{Introduction}

\section{Method}

\subsection{Dataset}

\section{Discussion}

\section{Figures}

\renewcommand\refname{Bibliography}

\bibliographystyle{ieeetr}  
\bibliography

\end{document}